\documentclass[]{article}

\usepackage[left=2.00cm, right=2.00cm, top=2.00cm, bottom=2.00cm]{geometry}
\usepackage[spanish,es-noshorthands]{babel}
\usepackage[utf8]{inputenc} % para tildes y ñ
\usepackage{graphicx} % para las figuras
\usepackage{xcolor}
\usepackage{listings} % para el código fuente en c++

\lstdefinestyle{customc}{
  belowcaptionskip=1\baselineskip,
  breaklines=true,
  frame=single,
  xleftmargin=\parindent,
  language=C++,
  showstringspaces=false,
  basicstyle=\footnotesize\ttfamily,
  keywordstyle=\bfseries\color{green!40!black},
  commentstyle=\itshape\color{gray!40!gray},
  identifierstyle=\color{black},
  stringstyle=\color{orange},
}
\lstset{style=customc}


%opening
\title{Práctica 1. Algoritmos devoradores}
\author{Miguel Cabral Ramirez \\ % mantenga las dos barras al final de la línea y este comentario
miguel.cabralramirez@alum.uca.es \\ % mantenga las dos barras al final de la linea y este comentario
Teléfono: xxxxxxxx \\ % mantenga las dos barras al final de la línea y este comentario
NIF: 32086649Q \\ % mantenga las dos barras al final de la línea y este comentario
}


\begin{document}

\maketitle

%\begin{abstract}
%\end{abstract}

% Ejemplo de ecuación a trozos
%
%$f(i,j)=\left\{ 
%  \begin{array}{lcr}
%      i + j & si & i < j \\ % caso 1
%      i + 7 & si & i = 1 \\ % caso 2
%      2 & si & i \geq j     % caso 3
%  \end{array}
%\right.$

\begin{enumerate}
\item Describa a continuación la función diseñada para otorgar un determinado valor a cada una de las celdas del terreno de batalla para el caso del centro de extracción de minerales. 

Escriba aquí su respuesta al ejercicio 1.

\item Diseñe una función de factibilidad explicita y descríbala a continuación.

Escriba aquí su respuesta al ejercicio 2.

\item A partir de las funciones definidas en los ejercicios anteriores diseñe un algoritmo voraz que resuelva el problema para el caso del centro de extracción de minerales. Incluya a continuación el código fuente relevante. 

\begin{lstlisting}
// sustituya este codigo por su respuesta
void placeDefenses(...) {

    List<Defense*>::iterator currentDefense = defenses.begin();
    while(currentDefense != defenses.end() && maxAttemps > 0) {

        (*currentDefense)->position.x = ((int)(_RAND2(nCellsWidth))) * cellWidth + cellWidth * 0.5f;
        ...
        ++currentDefense;
    }
}
\end{lstlisting}

\item Comente las características que lo identifican como perteneciente al esquema de los algoritmos voraces. 

\begin{lstlisting}
// sustituya este codigo por su respuesta
void selectDefenses(...) {

    unsigned int cost = 0;
    std::list<Defense*>::iterator it = defenses.begin();
    while(it != defenses.end()) {
        if(cost + (*it)->cost <= ases) {
            selectedIDs.push_back((*it)->id);
            cost += (*it)->cost;
        }
        ++it;
    }
}
\end{lstlisting}

\item Describa a continuación la función diseñada para otorgar un determinado valor a cada una de las celdas del terreno de batalla para el caso del resto de defensas. Suponga que el valor otorgado a una celda no puede verse afectado por la colocación de una de estas defensas en el campo de batalla. Dicho de otra forma, no es posible modificar el valor otorgado a una celda una vez que se haya colocado una de estas defensas. Evidentemente, el valor de una celda sí que puede verse afectado por la ubicación del centro de extracción de minerales.

Escriba aquí su respuesta al ejercicio 5.

\item A partir de las funciones definidas en los ejercicios anteriores diseñe un algoritmo voraz que resuelva el problema global. Este algoritmo puede estar formado por uno o dos algoritmos voraces independientes, ejecutados uno a continuación del otro. Incluya a continuación el código fuente relevante que no haya incluido ya como respuesta al ejercicio 3. 

\begin{lstlisting}
// sustituya este codigo por su respuesta
void placeDefenses(...) {

    List<Defense*>::iterator currentDefense = defenses.begin();
    while(currentDefense != defenses.end() && maxAttemps > 0) {

        (*currentDefense)->position.x = ((int)(_RAND2(nCellsWidth))) * cellWidth + cellWidth * 0.5f;
        ...
        ++currentDefense;
    }
}
\end{lstlisting}

\end{enumerate}

Todo el material incluido en esta memoria y en los ficheros asociados es de mi autoría o ha sido facilitado por los profesores de la asignatura. Haciendo entrega de este documento confirmo que he leído la normativa de la asignatura, incluido el punto que respecta al uso de material no original.

\end{document}
