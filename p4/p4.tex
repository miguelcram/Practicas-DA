\documentclass[]{article}

\usepackage[left=2.00cm, right=2.00cm, top=2.00cm, bottom=2.00cm]{geometry}
\usepackage[spanish,es-noshorthands]{babel}
\usepackage[utf8]{inputenc} % para tildes y ñ
\usepackage{graphicx} % para las figuras
\usepackage{xcolor}
\usepackage{listings} % para el código fuente en c++

\lstdefinestyle{customc}{
  belowcaptionskip=1\baselineskip,
  breaklines=true,
  frame=single,
  xleftmargin=\parindent,
  language=C++,
  showstringspaces=false,
  basicstyle=\footnotesize\ttfamily,
  keywordstyle=\bfseries\color{green!40!black},
  commentstyle=\itshape\color{gray!40!gray},
  identifierstyle=\color{black},
  stringstyle=\color{orange},
}
\lstset{style=customc}

%opening
\title{Práctica 4. Exploración de grafos}
\author{Miguel Cabral Ramirez \\ % mantenga las dos barras al final de la línea y este comentario
miguel.cabralramirez@alum.uca.es \\ % mantenga las dos barras al final de la linea y este comentario
Teléfono: xxxxxxxx \\ % mantenga las dos barras al final de la línea y este comentario
NIF: 32086649Q \\ % mantenga las dos barras al final de la línea y este comentario
}


\begin{document}

\maketitle

%\begin{abstract}
%\end{abstract}

% Ejemplo de ecuación a trozos
%
%$f(i,j)=\left\{ 
%  \begin{array}{lcr}
%      i + j & si & i < j \\ % caso 1
%      i + 7 & si & i = 1 \\ % caso 2
%      2 & si & i \geq j     % caso 3
%  \end{array}
%\right.$

\begin{enumerate}
\item Comente el funcionamiento del algoritmo y describa las estructuras necesarias para llevar a cabo su implementación.

Escriba aquí su respuesta al ejercicio 1.

\item Incluya a continuación el código fuente relevante del algoritmo.

Escriba aquí su respuesta al ejercicio 2.


\end{enumerate}

Todo el material incluido en esta memoria y en los ficheros asociados es de mi autoría o ha sido facilitado por los profesores de la asignatura. Haciendo entrega de esta práctica confirmo que he leído la normativa de la asignatura, incluido el punto que respecta al uso de material no original.

\end{document}
